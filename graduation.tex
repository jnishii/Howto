\documentclass[11pt]{jarticle}

%\usepackage{kanjifonts}
\usepackage{amsmath}
\usepackage{ascmac}
\usepackage{geometry}
\usepackage{fancyhdr}
\geometry{body={160mm,235mm}}

%% \renewcommand{\labelitemi}{一.}
%% \renewcommand{\labelitemi}{□ }
%% \renewcommand{\labelitemii}{□ }
\usepackage{latexsym}
\renewcommand{\labelitemi}{$\Box$}
\renewcommand{\labelitemii}{$\Box$}
\renewcommand{\labelitemiii}{- }


\begin{document}

\pagestyle{empty}

\begin{center}
  {\LARGE 卒業にあたって}\\
  {\Large 〜大学を去るヒトも去らないヒトも読みましょう〜}
\end{center}
\begin{flushright}
%{\popfamily\today 版}
{\today 版}
\end{flushright}
%\rule{\linewidth}{0.3pt} 

\begin{itemize}
\item 卒論/修論について
  \begin{itemize}
  \item 判定
  \begin{quote}
    判定会議の結果によっては,再発表,卒業延期等の判定が下されることもあります。
    発表が終わっても,すぐ行方不明になったりしないようにしましょう。s
  \end{quote}
  \item 卒論/修論修正版提出
  \begin{itemize}
  \item プレゼン資料等で作成した図表や内容を反映した完成版を作ってください
  \item \textbf{指定された日までに修正版を提出}しましょう
  \end{itemize}
  \end{itemize}
\item 春休みの過ごし方
  \begin{itemize}
  \item 春休みも研究室の予定は確認してください。
    来年もいる方は楽しいお勉強会があります。
  \item  行方不明になると、事務連絡等でたまに困る場合があります。
    (予想外に卒業単位不足が判明するヒトも毎年学部に一人くらいいます)。
    どっかに長期間行ってる場合には、その期間と、できれば連絡先を教えて
    下さい。
  \end{itemize}

\item 以下をCD-RやDVD等にいれて卒論・修論に添付してください。
  \begin{itemize}
  \item 卒論・修論で作ったプログラム・主要データ
    \begin{quote}
      プログラム中には、はじめて見る人にもわかりやすいようコメントをいれておいてください。
    \end{quote}
  \item 卒論・修論で作ったプログラムの説明
    \begin{quote}
      何をするプログラムか、その使いかた等をまとめ、READMEという名前
      のファイルにまとめておいてください。
    \end{quote}
  \item 卒論・修論で作った主要データの説明
    \begin{quote}
      何を表すデータか要点を書いて、READMEという名前のファイルにまとめ
      ておいてください。 
    \end{quote}
  \item 卒論・修論のドキュメントファイル(tex,図表,pdf)
  \item 卒論・修論のプレゼンテーション用ファイル
  \item 写真は研究室のギャラリーにも置いてください
  \item 動画像はギャラリーもしくは\verb|/home/public/Movie/|にもおきま
    しょう
  \item 論文は,バインダではさんだ提出形式の他,製本したもの2部を
    作成してください。
  \item 論文と発表資料は裏ページから見れるようにしておきましょう。
  \item 上記ファイルはHD上にも残しておいてください。学会発表等で使わせて
    頂くこともありますが、どうぞ御了承下さい。(主要なデータとして発表す
    るときには、連名にさせていただきます)
  \end{itemize}
\item 学会発表資料
  \begin{itemize}
  \item 学会発表を行ったヒトは,関連資料を\verb|/home/public|において
    下さい。詳しくは裏ページを見てください。
  \end{itemize}
\item 身の回りの片付け
  \begin{itemize}
  \item いなくなるヒトも、まだいるヒトも、卒業までに机・本棚の整理をしましょう。
  \item HDのファイルも掃除しましょう。
    \begin{itemize}
    \item 大事なファイルはCD-RやDVDに焼いておくとどこにでも持っていけ
      て便利です。
    \item 上に書いたように研究関係のファイルは整理して残しておいてください。
    \item 研究関係のいろいろなデータファイルも説明を書いたREADMEととも
      に残しておいて下さい。
    \end{itemize}
  \end{itemize}
\item 裏ページの「研究室での生活」も見ておいてください。
\item その他,卒業までにするべきことがないか西井にご確認ください。
\end{itemize}

\vspace{5mm}
\begin{center}
  {\LARGE 研究室を去る方へ}
\end{center}

\begin{itemize}
\item 研究室用メール(@bcl.sci.yamaguchi-u.ac.jp)の設定
  \begin{itemize}
  \item 見られちゃまずいファイルは、HDから消しときましょう。
  \item 研究室のメールアドレスで加入していたメーリングリストは脱会しましょう。
  \item 卒業後も使いたい方は申し出てください。申し出がなければ卒業
  後にメール着信拒否の設定をします。
  \item 必要であればメールのフォワード設定をしてください
  (\verb|~/.forward|というファイルに転送先メールアドレスを書いておく)。 
  \end{itemize}
\item 大学を去るまでにカードキーを返却してください。
\item 今後の連絡先・就職先等が、わかり次第教えて下さい。
\item 卒業後もたまには消息を御連絡下さい。
  特に住所変更等ありましたら御連絡下さい。 
\end{itemize}

\begin{center}
  {\large 社会にて:世の中の掟}
\end{center}

世の中は人と人の世界です。
ゴマをする必要はありませんが、
働くようになってからは、何かしていただいたことには、必ずお礼をしましょう。
縁を大切にすると、世の中が広がります。
\begin{itemize}
\item 挨拶は気持ちよく.
\item 日頃御世話になっている方には,最低限年賀状は欠かさずに.
\item 特に,社会に出てから御世話になる機会があった人には、お礼状や御中元・御歳暮を.
\end{itemize}

\vspace*{1cm}
\begin{center}
  {\large さいごに}
\end{center}
\begin{itemize}
\item 元気に楽しくお過ごし下さい。
\end{itemize}

\end{document}