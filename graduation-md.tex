\documentclass[11pt, ]{jsarticle}
\usepackage[hypertex]{hyperref}
\usepackage[dvipdfmx]{color}
\usepackage{listings,jlisting}
\newcommand{\passthrough}[1]{#1}
\usepackage{amsmath}
\usepackage{ascmac}
\usepackage[width=16cm, height=23cm]{geometry}
%\usepackage[height=24cm,width=16cm]{geometry}
\usepackage{fancyhdr}

\renewcommand{\labelenumi}{\arabic{enumi})}
\renewcommand{\labelenumii}{\alph{enumii})}

\providecommand{\tightlist}{%
   \setlength{\itemsep}{0pt}\setlength{\parskip}{0pt}}

\begin{document}
\renewcommand{\lstlistingname}{リスト}
\lstset{language = c,
numbers = left,
numberstyle = {\footnotesize \emph},
numbersep = 10pt,
breaklines = true,
breakindent = 40pt,
frame = tlRB,
frameround = ffft,
framesep = 3pt,
rulesep = 1pt,
rulecolor = {\color{black}},
rulesepcolor = {\color{black}},
flexiblecolumns = false,
keepspaces = true,
basicstyle = \footnotesize,
identifierstyle = \footnotesize,
%commentstyle = \fontfamily{ptm}\selectfont\footnotesize,
commentstyle = \selectfont\footnotesize,
stringstyle = \scshape\footnotesize,
tabsize = 4,
% lineskip = 0.1ex
}


\noindent
%\rule{\linewidth}{0.3pt}
\begin{center}
  \textbf{\LARGE 卒業にあたって}
\end{center}
%\rule{\linewidth}{0.3pt}
  

\hypertarget{ux5927ux5b66ux3092ux53bbux308bux30d2ux30c8ux3082ux53bbux3089ux306aux3044ux30d2ux30c8ux3082ux8aadux307fux307eux3057ux3087ux3046}{%
\section{〜大学を去るヒトも去らないヒトも読みましょう〜}\label{ux5927ux5b66ux3092ux53bbux308bux30d2ux30c8ux3082ux53bbux3089ux306aux3044ux30d2ux30c8ux3082ux8aadux307fux307eux3057ux3087ux3046}}

\hypertarget{ux5352ux8ad6ux4feeux8ad6ux306eux4feeux6b63ux7248ux63d0ux51faux306bux3064ux3044ux3066}{%
\subsection{卒論/修論の修正版提出について}\label{ux5352ux8ad6ux4feeux8ad6ux306eux4feeux6b63ux7248ux63d0ux51faux306bux3064ux3044ux3066}}

\begin{itemize}
\tightlist
\item
  プレゼン資料等で作成した図表や内容を反映した完成版を作ってください。
\item
  \textbf{指定された日までに修正版を提出}しましょう。
\item
  論文は,両面コピーをして製本したもの3部(本人保存版含む)を作成してください。
\end{itemize}

\hypertarget{ux5352ux8ad6ux4feeux8ad6ux306eux5224ux5b9aux306bux3064ux3044ux3066}{%
\subsection{卒論/修論の判定について}\label{ux5352ux8ad6ux4feeux8ad6ux306eux5224ux5b9aux306bux3064ux3044ux3066}}

\begin{itemize}
\tightlist
\item
  2月末に一旦仮成績が発表されますが,卒論/修論修正が不十分な場合等は3月にある判定会議で,成績判定が取り消されることもあります。しっかり修正しましょう。
\end{itemize}

\hypertarget{ux6625ux4f11ux307fux306eux904eux3054ux3057ux65b9}{%
\subsection{春休みの過ごし方}\label{ux6625ux4f11ux307fux306eux904eux3054ux3057ux65b9}}

\begin{itemize}
\tightlist
\item
  春休みも研究室の予定は確認してください。春以降もいる方は楽しいお勉強会があります。春にいなくなる人対象の行事もあります。
\item
  行方不明にならないようにしましょう。予想外に卒業単位不足が判明するヒトも毎年学科に一人くらいいて,連絡に困ることがあります。どこかに長期間行く場合には、その期間とできれば連絡先を教えて下さい。
\end{itemize}

\hypertarget{ux5352ux8ad6ux4feeux8ad6ux5f8cux5352ux696dux5f0fux307eux3067ux306bux3059ux308bux3053ux3068}{%
\subsection{卒論/修論後,卒業式までにすること}\label{ux5352ux8ad6ux4feeux8ad6ux5f8cux5352ux696dux5f0fux307eux3067ux306bux3059ux308bux3053ux3068}}

\hypertarget{ux7814ux7a76ux30c7ux30fcux30bfux306eux3068ux308aux307eux3068ux3081}{%
\subsubsection{研究データのとりまとめ}\label{ux7814ux7a76ux30c7ux30fcux30bfux306eux3068ux308aux307eux3068ux3081}}

以下を卒論・修論とともに以下のようにまとめてください。

\begin{itemize}
\tightlist
\item
  以下はGitHubにおいてください。GitHubの容量制限にひっかかるときには西井に相談してください。

  \begin{itemize}
  \tightlist
  \item
    卒論・修論で作ったプログラム・主要データ

    \begin{itemize}
    \tightlist
    \item
      プログラム中には、初めて見る人にもわかりやすいようコメントをいれてください。
    \end{itemize}
  \item
    卒論・修論で作ったプログラムの説明

    \begin{itemize}
    \tightlist
    \item
      各プログラムの処理内容と使い方の簡単な説明をREADME.md
      ファイルに,マークダウン方式でまとめておいてください。
    \end{itemize}
  \item
    卒論・修論で作った主要データの説明

    \begin{itemize}
    \tightlist
    \item
      何を表すデータかをまとめたREADME.mdにまとめてください。
    \end{itemize}
  \item
    卒論・修論のドキュメントファイル(tex,図表,pdf)
  \item
    卒論・修論のプレゼンテーション用ファイル
  \end{itemize}
\item
  論文とプレゼンのファイルは裏ページのレポジトリ(bcl-group/theses)にもおいてください。
\item
  以下はheraにおいてください。置き場所は西井に相談してください。

  \begin{itemize}
  \tightlist
  \item
    発表やプレゼンには使わなかったが,研究に関連する画像や動画。
  \end{itemize}
\item
  上記ファイル学会発表等で使わせて頂くこともありますが、どうぞ御了承下さい。(主要なデータとして発表するときには、連名にさせていただきます)
\end{itemize}

\hypertarget{ux305dux306eux4ed6ux306eux7247ux4ed8ux3051}{%
\subsubsection{その他の片付け}\label{ux305dux306eux4ed6ux306eux7247ux4ed8ux3051}}

\begin{itemize}
\tightlist
\item
  学会発表資料

  \begin{itemize}
  \tightlist
  \item
    学会発表を行ったヒトは,関連資料を卒論・修論と同様にgithubもしくは\passthrough{\lstinline!/home/public!}において下さい。詳しくは裏ページを見てください。
  \end{itemize}
\item
  身の回りの片付け

  \begin{itemize}
  \tightlist
  \item
    いなくなるヒトも、まだいるヒトも、卒業までに机・本棚の整理をしましょう。
  \item
    ユーザホームや'/home2'にあるファイルも掃除しましょう。

    \begin{itemize}
    \tightlist
    \item
      大事なファイルはDVDに焼いておくとどこにでも持っていけて便利です。
    \item
      上に書いたように研究関係のファイルは整理して残しておいてください。
    \item
      研究関係のいろいろなデータファイルも説明を書いたREADMEととも
      に残しておいて下さい。
    \end{itemize}
  \end{itemize}
\item
  裏ページの「研究室での生活」も見ておいてください。
\item
  その他,卒業までにするべきことがないか西井にご確認ください。
\end{itemize}

\hypertarget{ux7814ux7a76ux5ba4ux3092ux53bbux308bux65b9ux3078}{%
\section{研究室を去る方へ}\label{ux7814ux7a76ux5ba4ux3092ux53bbux308bux65b9ux3078}}

\begin{itemize}
\tightlist
\item
  メール(@cc.yamaguchi-u.ac.jp)の設定

  \begin{itemize}
  \tightlist
  \item
    ホームにメールを保存している人は消しておきましょう。
  \item
    大学のメールアドレスで加入していたメーリングリストは脱会しましょう。
  \end{itemize}
\item
  大学を去るまでにカードキーを返却してください。
\item
  今後の連絡先・就職先等が、わかり次第教えて下さい。
\item
  卒業後もたまには消息を御連絡下さい。
  特に住所変更等ありましたら御連絡下さい。
\end{itemize}

\hypertarget{ux793eux4f1aux306bux3066ux4e16ux306eux4e2dux306eux639f}{%
\subsection{社会にて:世の中の掟}\label{ux793eux4f1aux306bux3066ux4e16ux306eux4e2dux306eux639f}}

世の中は人と人の世界です。
ゴマをする必要はありませんが、働くようになってからは、何かしていただいたことには、必ずお礼をしましょう。縁を大切にすると、世の中が広がります。

\begin{itemize}
\tightlist
\item
  挨拶は気持ちよく
\item
  日頃御世話になっている方には,最低限年賀状は欠かさずに
\item
  特に,社会に出てから御世話になる機会があった人には、お礼状や御中元・御歳暮を
\end{itemize}

\hypertarget{ux3055ux3044ux3054ux306b}{%
\subsection{さいごに}\label{ux3055ux3044ux3054ux306b}}

\begin{itemize}
\tightlist
\item
  元気に楽しくお過ごし下さい。
\end{itemize}

\end{document}
