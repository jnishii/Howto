\documentclass[11pt, ]{jsarticle}
\usepackage[hypertex]{hyperref}
\usepackage{amsmath}
\usepackage{ascmac}
%\usepackage[height=24cm,width=16cm]{geometry}
\usepackage{fancyhdr}

\renewcommand{\labelenumi}{\arabic{enumi})}
\renewcommand{\labelenumii}{\alph{enumii})}

\providecommand{\tightlist}{%
   \setlength{\itemsep}{0pt}\setlength{\parskip}{0pt}}

\begin{document}


\noindent
%\rule{\linewidth}{0.3pt}
\begin{center}
  \textbf{\LARGE 研究室での暮らし}
\end{center}
%\rule{\linewidth}{0.3pt}
  

\begin{itemize}
\tightlist
\item
  基本

  \begin{itemize}
  \tightlist
  \item
    挨拶は大きな声で気持ちよく
  \item
    遅刻,〆切等に遅れる時には,事前に連絡
  \item
    やみくもに人にいろいろ聞かず,まず自分で調べる
  \item
    聞くべきことは聞く
  \item
    やるべきことはやる
  \item
    困ったことがあれば言う
  \item
    西井は物忘れが得意であることを覚えておく
  \end{itemize}
\item
  重要

  \begin{itemize}
  \tightlist
  \item
    学研災(付帯賠償も)に必ず加入。共通教育の学務係で受付。
  \item
    \textbf{注意}:研究室所属後の生活は、普通の社会生活に準じます。
    講義の夏季休暇中も研究室の活動はあります。休暇は社会生活の常識の範囲内に。
  \end{itemize}
\item
  毎日の生活

  \begin{itemize}
  \tightlist
  \item
    \textbf{朝は何もなくても9時半までにくる}
  \item
    \textbf{早くても17時までは研究をがんばる}
  \item
    セミナー・本読み等は必ず出席する。\textbf{欠席のときには必ず連絡}
  \item
    用事で研究室に来れない日や時間帯は早めにGoogle Calendarに登録
  \item
    \textbf{昼間のアルバイトは原則として禁止}。用事がある場合も,\textbf{セミナー・本読みや卒論・修論に支障がないように}
  \item
    日頃の出欠や取組状況は特別研究,情報科学講究I,IIに反映されます。また,就職活動の際の大学推薦の推薦状にも反映されます。
  \end{itemize}
\item
  研究室での生活

  \begin{itemize}
  \tightlist
  \item
    帰るときには (原則として)PCとディスプレイの電源をおとす
  \item
    部屋を最後に出るときには\ldots{}

    \begin{itemize}
    \tightlist
    \item
      \textbf{火元確認}
    \item
      必ず\textbf{鍵をかける}(前後)
    \item
      \textbf{窓を閉める}
    \end{itemize}
  \item
    お知らせ(ホワイトボード、メール,Slack)は毎日チェック
  \item
    いろいろこまめに連絡する(SlackでOK)

    \begin{itemize}
    \tightlist
    \item
      新しい物品が届いたとき(保証書や請求書が入ってれば西井に渡す)
    \item
      消耗品が減ったとき(プリンタのトナー等)
    \item
      機器の故障
    \end{itemize}
  \item
    研究室はみんなで協力して清潔に
  \item
    研究室備品は規定の場所にしまう
  \item
    上級生は下級生の面倒を見る。
  \end{itemize}
\item
  研究室のホームページについて

  \begin{itemize}
  \tightlist
  \item
    修正に御協力ください
  \end{itemize}
\item
  研究室の裏ページ(https://github.org/bcl-group/memo/)について

  \begin{itemize}
  \tightlist
  \item
    誰でもいじれます。情報追加にご協力を。
  \end{itemize}
\item
  ギャラリーも活用してください
\item
  Slackについて

  \begin{itemize}
  \tightlist
  \item
    きちんとチェックして,質問等にはお返事をすみやかに

    \begin{itemize}
    \tightlist
    \item
      研究室のことや,研究に関すること等,いろいろな記録を残してください

      \begin{itemize}
      \tightlist
      \item
        ex1)
        「バックアップ用プログラムを修正して,/home/venus/admin/\ldots{}
        においた」
      \item
        ex2) 「pythonの使い方をhttp://\ldots で勉強した」
      \end{itemize}
    \item
      サーバ等の設定の際には,何をしたか細かく記録して(うまくいかないことも含めて),あとで整理した裏ページに貼り付ける
    \end{itemize}
  \end{itemize}
\item
  研究報告セミナーの趣旨

  \begin{itemize}
  \tightlist
  \item
    \textbf{重要}
    セミナーの目的は\textbf{良い結果を説明することではありません}。良い結果が出たら喜ばしいことですが,本当に信頼できる結果を出す取り組みができているのか,研究が順調に進んでいないなら何が問題かを明らかにすることが目的です。何をしてどういう状況になっているか具体的に説明して下さい。
  \end{itemize}
\item
  研究について

  \begin{itemize}
  \tightlist
  \item
    自主的にどんどん進める
  \item
    失敗を怖がらない
  \item
    予想通りの結果にならなくても,率直に報告する。ミスに気づくきっかけになればめでたい。予想外の結果がかえって発見につながることもある。
  \item
    不正(捏造やコピペ)は絶対しない
  \end{itemize}
\item
  研究発表等について

  \begin{itemize}
  \tightlist
  \item
    学外で研究発表を行うときは,旅費の援助は出来る限りします。
    ただし、発表者以外への援助は原則として難しいことは御了承下さい。
  \item
    研究発表原稿は、ゆとりをもって西井に見せて下さい。
  \item
    研究発表における著者の順序は西井が決めます。
  \end{itemize}
\item
  さいごに

  \begin{itemize}
  \tightlist
  \item
    \textbf{報告・連絡・相談}をきっちりと。
  \end{itemize}
\end{itemize}

\end{document}
