\documentclass[12pt, ]{jsarticle}
\usepackage[hypertex]{hyperref}
\usepackage[dvipdfmx]{color}
\usepackage{listings,jlisting}
\usepackage{amsmath}
\usepackage{ascmac}
%\usepackage[height=24cm,width=16cm]{geometry}
\usepackage{fancyhdr}

\renewcommand{\labelenumi}{\arabic{enumi})}
\renewcommand{\labelenumii}{\alph{enumii})}

\providecommand{\tightlist}{%
   \setlength{\itemsep}{0pt}\setlength{\parskip}{0pt}}

\begin{document}
\renewcommand{\lstlistingname}{リスト}
\lstset{language = c,
numbers = left,
numberstyle = {\footnotesize \emph},
numbersep = 10pt,
breaklines = true,
breakindent = 40pt,
frame = tlRB,
frameround = ffft,
framesep = 3pt,
rulesep = 1pt,
rulecolor = {\color{black}},
rulesepcolor = {\color{black}},
flexiblecolumns = false,
keepspaces = true,
basicstyle = \footnotesize,
identifierstyle = \footnotesize,
%commentstyle = \fontfamily{ptm}\selectfont\footnotesize,
commentstyle = \selectfont\footnotesize,
stringstyle = \scshape\footnotesize,
tabsize = 4,
% lineskip = 0.1ex
}
\newcommand{\passthrough}[1]{#1}



\noindent
\rule{\linewidth}{0.3pt}
\begin{center}
  \textbf{\LARGE 研究発表の掟}
\end{center}
\begin{flushright}
\today 版\\
西井 淳
\end{flushright}
\rule{\linewidth}{0.3pt}
    {\small
  \begin{quote}
        \hypersetup{linkcolor=black}
      \setcounter{tocdepth}{3}
      \tableofcontents
    \end{quote}}
  %]
          \rule{\linewidth}{0.3pt}
      

\hypertarget{ux306fux3058ux3081ux306b}{%
\section{はじめに}\label{ux306fux3058ux3081ux306b}}

発表用のスライドの作成の基本の多くは、論文を書く場合と同じである。
まず「論文の書き方の掟」をよく読むこと。

\hypertarget{ux767aux8868ux306eux639f}{%
\section{発表の掟}\label{ux767aux8868ux306eux639f}}

研究発表の基本は,自分がしたことを列挙することではない。
\textbf{「自分のしたことを他人に興味をもって聞いてもらえるように誠意をつくすこと」}である。

\begin{enumerate}
\item
  聞き手の方を向いて話す.

  \begin{quote}
  間違えても,お尻をむけたまま話し続けるよう
  な失礼なことをしてはいけない。
  下半身を観客の方に向けておくのをホームポジションとすると,スライドを指さしたりしても上体はすぐに観客の方に戻る。
  \end{quote}
\item
  聞き手の視線をさえぎらないように,居場所を考える。

  \begin{quote}
  スクリーンの前にたちはだかって,映している資料を隠してはいけない。
  \end{quote}
\item
  部屋の一番後の壁に語りかけるように。

  \begin{quote}
  部屋の隅に十分伝わる話し方をすれば,その部屋の全員に伝わる。
  部屋の大きさや響具合で,声の大きさや話のテンポは決まる。
  \end{quote}
\item
  何の話をしているか,その目的を常に明らかに。

  \begin{quote}
  必ず大きな目的を言ってから,小さな話にうつる。
  \end{quote}
\item
  常に論理的展開に注意。「つっこみ」の入らないような話し方を

  \begin{quote}
  観客は漫才をしにきてるのではない。
  「なんでやねん」と思われるような話し方をしてはいけない。
  (「なんでやねん」と思わせてから,すかさず自分でも「つっこみ」
  を入れて聞き手を納得させる中級者以上むけの技もあるが,特に印象づけた
  い点のみに留めるのが吉)
  \end{quote}
\item
  話し過ぎない

  \begin{quote}
  一度にたくさん話しても聞いている方は混乱するだけである。
  自分のしたことを全て話す必要はない。
  自分の言いたいことは何かをよく考え,必要最小限のことを,しっかり伝える。
  \end{quote}
\item
  しっかり話す

  \begin{quote}
  話し手にとって必要最小限と思える話し方は、聞き手にとってはわかりに
  くくなる可能性が高い。言いたいことを伝えるための方法をよく考え
  る。一般的(抽象的な)な表現が伝えたいことの中にあるときには、必ず具
  体例も考える。具体例を考えることは、自分の理解を深めることにもなる。
  \end{quote}
\item
  構成に注意

  \begin{quote}
  自分が研究をしたその順番と同じように発表の内容の順番も決めたくなる
  ことが多いが,全然違う順番にしたほうが聞き手にはわかりやすい場合がある。
  \end{quote}
\item
  つなぎめに注意

  \begin{quote}
  スライドやOHPを次のページにうつるとき,前のページと次のページの関
  係がわかるような一言を考えること。つなぎを上手にしないと,全体のス
  トーリーが伝わらない。
  \end{quote}
\item
  レーザポインタの動きは最小限に

  \begin{quote}
  レーザポインタはぐるぐる動かしたりしない。一点をピシッと指す。
  \end{quote}
\item
  「えと」「あの」等、無駄なことは言わない。
  手で何かをいじる等無駄な動きはしない。

  \begin{quote}
  聞き手に対する無駄な信号は、意思伝達のノイズとなる。
  ノイズは必ずしも意識にのぼるものではないが、
  意思伝達を阻害する要因になる。
  \end{quote}
\item
  発表練習は口に出して100回する。

  \begin{quote}
  頭の中だけの練習と声も出す練習では脳の状態が大きく変わる。
  頭の中だけの練習だけでは,本番に口が動かなくなる。
  必ず声を出して,何度も練習する。
  \end{quote}
\item
  原稿を読みながらの発表は不可。

  \begin{quote}
  原稿の棒読みは,聞き手には非常に理解しづらい。
  そもそも自分でやったことを話すのだから,原稿がないと話せないと
  いうのは馬鹿げた話で,不信感を抱かれる場合もある。
  自分でストーリーを思いだしながら話せるようによく練習すること。
  (100回も練習すりゃ,普通覚えるけどね)。
  逆に,話す内容が思い出しやすいようにプレゼン資料をよく工夫すること。
  \end{quote}
\item
  友達に「悪いところ」を指摘してもらう。

  \begin{quote}
  社会にでると,「おせじ」は言ってもらえても「悪いところ」を言っても
  らえる機会はほとんどない。学生のうちにしっかり指摘してもらう。
  ただし,指摘してもらったことは有難く受け止め,ちゃんとお礼を言う。
  \end{quote}
\end{enumerate}

\hypertarget{ux767aux8868ux306eux5965ux6280}{%
\section{発表の奥技}\label{ux767aux8868ux306eux5965ux6280}}

自分の話してることがいかに面白いことなのか、その感動を伝えましょう!

\begin{itemize}
\item
  かしこまった話し方にならない。
  友達に感動したことを普段話すような話し方を少しだけ丁寧にするのがベスト!
\item
  できれば笑いをとるポイントをつくる。
  ただし、ふざけた雰囲気にしてはいけない。また、笑いをとっても肝心の発
  表がしょぼいとかなりみっともないので要注意。
\end{itemize}

\hypertarget{ux767aux8868ux69cbux6210ux306eux639f}{%
\section{発表構成の掟}\label{ux767aux8868ux69cbux6210ux306eux639f}}

発表の大きな流れは以下の通り。

\begin{enumerate}
\item
  タイトルと名前

  \begin{itemize}
  \tightlist
  \item
    自己紹介くらいはしときましょ。
  \end{itemize}
\item
  はじめに

  \begin{itemize}
  \item
    \textbf{研究の背景}(世の中でされてる研究等), \textbf{問題提起},
    \textbf{研究の目的}について述べる.
  \item
    \textbf{研究の存在意義が問われる部分}である. 聞き手の興味を引
    き付けるよう工夫すること。
  \end{itemize}
\item
  発表の流れ

  \begin{itemize}
  \tightlist
  \item
    話の全体の流れをわかってもらうために,発表のおおまかな構成(目次に相当)を書いたスライドをここで出すことが多い。
  \item
    「発表の流れ」を示すスライドは,発表の進行状況を示すために何度も提示することもある。
  \end{itemize}
\item
  手法

  \begin{itemize}
  \tightlist
  \item
    実験手法や理論的準備等を具体的に述べる.
  \end{itemize}
\item
  結果・考察

  \begin{itemize}
  \item
    得られた結果と考察を書く.
  \item
    はじめに述べた\textbf{目的に対する結果および考察}をきちんと述べること。
  \end{itemize}
\item
  まとめ

  \begin{itemize}
  \tightlist
  \item
    結局何をしてどういう結果になったのか,はじめに述べた
    \textbf{目的に対してどういう結果が得られたか}を,手短にまとめる。
    つまり、目的には、「はじめに」と「結果・考察」そしてここの「まとめ」
    の合計三回ふれることになる。
  \end{itemize}
\item
  今後の課題

  \begin{itemize}
  \tightlist
  \item
    残された課題を具体的に述べる。
  \end{itemize}
\item
  謝辞(オプショナル)

  \begin{itemize}
  \tightlist
  \item
    最後に研究協力者の紹介をすることもある。
  \end{itemize}
\item
  おまけ(オプショナル)

  \begin{itemize}
  \tightlist
  \item
    最後にうけをとるためのスライドをつくる人もいるが,発表がいまい
    ちだとかえって墓穴をほってしまうことも多いので注意。
  \end{itemize}
\end{enumerate}

\hypertarget{ux767aux8868ux7528ux8cc7ux6599ux306eux66f8ux304dux65b9ux306eux639f}{%
\section{発表用資料の書き方の掟}\label{ux767aux8868ux7528ux8cc7ux6599ux306eux66f8ux304dux65b9ux306eux639f}}

発表用資料の作り方には,いろいろなお流儀がある。
ここで紹介するのは,その中でも,特に初心者におすすめのもの。
もっとも,上級と思われる人でも,このお流儀の人も多い。

\begin{itemize}
\item
  \textbf{いきなりパワーポイントに向かうのは厳禁!} \textgreater{}
  資料の作成に入る前に,話のストーリーを考え,どんな資料を用意すればよいか,その大雑把なページ構成をまず考える。
  \textgreater{}
  一枚の紙に,資料の各ページに相当する四角をならべて,タイトルや大雑把な構成を書き込むとよい。
  \textgreater{}
  ポスト・イット一枚をスライド一枚に見立てて下書きするのも良い。
  \textgreater{}
  発表時間を考え,内容および枚数の調整をしてから,いよいよ一枚一枚の作成に入る。
\item
  各ページに必ずタイトルをつける。

  \begin{quote}
  逆に言うと,一つのタイトルがつくような内容を一つのページにまとめるようにする。
  \end{quote}
\item
  平均1枚/分が原則。

  \begin{quote}
  もちろん内容によるが, 2枚/分はほとんどの場合不可能。
  仮に話すことが出来たとしても,聞いてる方がついてこれない。
  \end{quote}
\item
  図をたくさん。文字は少し。

  \begin{quote}
  言いたいことは全て絵やグラフで説明できるようにする。
  \end{quote}
\item
  言いたいことは箇条書等で簡潔に。

  \begin{quote}
  書いてあるキーワードをつないで読めば文章になるように書くのがコツ。
  ぶつぶつ話すことを口に出しながら作ると,話しやすい資料ができる。
  頭だけで考えようとしないこと。
  \end{quote}
\end{itemize}

\hypertarget{ux56f3ux8868ux306eux8aacux660eux306bux3064ux3044ux3066}{%
\section{図表の説明について}\label{ux56f3ux8868ux306eux8aacux660eux306bux3064ux3044ux3066}}

\begin{itemize}
\item
  結果のエッセンスを視覚的にわかりやすく示す上で,図表はとても重要。
  かしこまった図だけでなく,へのへのもへじのような簡単なイラストで
  もよいので,図で説明をする工夫をすること。
\item
  グラフを出したら,すぐに縦軸・横軸の説明をする。

  \begin{quote}
  初めて見るグラフの意味を理解するには少し時間がかかる。
  まずは軸の説明から丁寧に。

  \begin{enumerate}
  \item
    図表が\textbf{何を表すものか}を話す。
  \item
    縦軸・横軸が何か説明する。(あとの説明に含める場合もある)
  \item
    図表を\textbf{見てわかること}をそのまま話す。
  \item
    その理由についての\textbf{考察}を話す.
  \end{enumerate}

  \begin{quote}
  例)図に2月1日の気温の時間変化の様子を示す.
  (横軸が時刻,縦軸が気温である)。早朝には氷点下にまで気温が
  下がっているが日中の最高気温は3月並にまで上昇したことがわかる.
  この日は高気圧に覆われたため, 朝は放射冷却によって冷え込んだが,
  晴天によ り日中の気温が上昇したことが原因と考えられる.
  \end{quote}
  \end{quote}
\end{itemize}

\hypertarget{ux3053ux306eux30c9ux30adux30e5ux30e1ux30f3ux30c8ux306eux8457ux4f5cux6a29ux306bux3064ux3044ux3066}{%
\section{このドキュメントの著作権について}\label{ux3053ux306eux30c9ux30adux30e5ux30e1ux30f3ux30c8ux306eux8457ux4f5cux6a29ux306bux3064ux3044ux3066}}

\begin{enumerate}
\item
  本稿の著作権は西井淳\url{nishii@sci.yamaguchi-u.ac.jp}が有します。
\item
  非商用目的での複製は許可しますが、修正を加えた場合は必ず修正点および加筆者の
  氏名・連絡先、修正した日付を明記してください。また本著作権表示の削除は行っ
  てはいけません。
\item
  本稿に含まれている間違い等によりなんらかの被害を被ったとしても著者は一切
  責任を負いません。
\end{enumerate}

間違い等の連絡や加筆修正要望等の連絡は大歓迎です。

\end{document}
