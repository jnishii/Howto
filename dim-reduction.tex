\documentclass[a4paper,11pt]{jsarticle}

\usepackage{amsmath,amsfonts}
\usepackage{bm}
\usepackage[dvipdfmx]{graphicx}

\begin{document}

\title{時系列信号の次元圧縮}
\author{西井淳}
\date{\today}
\maketitle

\section{目的}
取得した時系列信号$\bm{e}(t)=(e_1(t),e_2(t),\ldots,e_N(t))^t$をできるだけ少ない基底$\{\bm{p}_1,\bm{p}_2,\ldots,\bm{p}_M\}$, $\bm{p}_i=(p_{i1},p_{i2},\ldots, p_{iN})^t$, $(i=1,2,\ldots,M)$で表す。

\begin{align}
\begin{pmatrix}
  e_1(t) \\
  e_2(t) \\
  \vdots\\
  e_N(t)
\end{pmatrix}
&=
c_1(t)
\begin{pmatrix}
p_{11}\\
p_{12}\\
\vdots\\
p_{1N}
\end{pmatrix}
+
c_2(t)
\begin{pmatrix}
p_{21}\\
p_{22}\\
\vdots\\
p_{2N}
\end{pmatrix}
+ \cdots
\end{align}
\begin{align}
  \bm{e}(t)&=c_1(t)\bm{p}_1+c_2(t)\bm{p}_2+\cdots
\end{align}
ここで,$c_i(t)$は基底$\{\bm{p}_i\}$を座標軸とする空間で$\bm{e}(t)$を表した場合の座標値を表す。
この基底は,主成分分析では主成分軸,因子分析では因子に対応する。各基底ベクトルは単位ベクトル($||\bm{p}_i||=1$)になるようにとる。

主成分分析では,
この基底ベクトルが主成分軸を表し,基底ベクトルの要素を主成分負荷量,係数$c_i$を第$i$主成分得点とよぶ。
因子分析では基底ベクトルの要素$p_{ij}$を因子負荷量,係数$c_i$を第$i$共通因子とよぶ。
\begin{align}
  \begin{pmatrix}
    e_1(t_1) & e_1(t_2) &\cdots\\
    e_2(t_1) & e_2(t_2) & \cdots\\
    \vdots&&\\
    e_N(t_1)& e_N(t_2) & \cdots\\
  \end{pmatrix}
  =
  \begin{pmatrix}
  p_{11} &p_{21}& \cdots\\
  p_{12} &p_{22}& \cdots\\
  \vdots &&\\
  p_{1N} &p_{2N}& \cdots\\
  \end{pmatrix}
  \begin{pmatrix}
    c_1(t_1) & c_1(t_2) & \cdots\\
    c_2(t_1) & c_2(t_2) & \cdots\\
    \vdots & & \\
    c_N(t_1) & c_1(t_2) & \cdots\\
    \end{pmatrix}
  \end{align}

  \section{主成分分析(Principle Component Analysis, PCA)}

変数の分布を考慮した座標軸を作る。

\begin{itemize}
\item 原点をデータ点の平均ベクトルにとる
\item 元データを射影した点の分散が大きくなる順に座標軸(主成分)を選択
\item 直交基底
\item 得られる座標軸は元データの次元と同じ。その中から累積寄与率に基づいて選択
\end{itemize}

\section{因子分析(Factor Analysis, FA)}

変数間の相関関係を因子によって説明する。

\begin{itemize}
  \item 原点をデータ点の平均ベクトルにとる。
  \item 基底の数は分析時に指定
  \item 基底の選び方はいろいろある
  \begin{itemize}
    \item 主因子法: 因子寄与が大きなものから順
    \item 最尤法: 尤度が最大になる順
    \item 最小自乗: 残差を小さくする順
    \item etc
  \end{itemize}
    \item 直交基底になるとは限らない
    \end{itemize}
  
\section{非負値因子分解(Nonnegative Matrix Factorization, NMF)}

\begin{itemize}
  \item 因子分析の一種
  \item 基底の要素および係数が正になるように座標軸を決定
  \item 残差を表現するVariance Accounted For (VAF)によって因子数を決定することが多い
\end{itemize}

\end{document}