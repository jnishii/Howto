\documentclass[12pt, ]{jsarticle}
\usepackage[hypertex]{hyperref}
\usepackage[dvipdfmx]{color}
\usepackage{listings,jlisting}
\usepackage{amsmath}
\usepackage{ascmac}
%\usepackage[height=24cm,width=16cm]{geometry}
\usepackage{fancyhdr}

\renewcommand{\labelenumi}{\arabic{enumi})}
\renewcommand{\labelenumii}{\alph{enumii})}

\providecommand{\tightlist}{%
   \setlength{\itemsep}{0pt}\setlength{\parskip}{0pt}}

\begin{document}

\renewcommand{\lstlistingname}{リスト}
\lstset{language = c,
numbers = left,
numberstyle = {\footnotesize \emph},
numbersep = 10pt,
breaklines = true,
breakindent = 40pt,
frame = tlRB,
frameround = ffft,
framesep = 3pt,
rulesep = 1pt,
rulecolor = {\color{black}},
rulesepcolor = {\color{black}},
flexiblecolumns = false,
keepspaces = true,
basicstyle = \footnotesize,
identifierstyle = \footnotesize,
%commentstyle = \fontfamily{ptm}\selectfont\footnotesize,
commentstyle = \selectfont\footnotesize,
stringstyle = \scshape\footnotesize,
tabsize = 4,
% lineskip = 0.1ex
}


\pagestyle{empty}
\rule{\linewidth}{0.3pt}
\begin{center}
  \textbf{\LARGE 論文の書き方の掟}
\end{center}
\begin{flushright}
\today 版\\
西井 淳
\end{flushright}
\rule{\linewidth}{0.3pt}
{\small
\begin{quote}
\begin{screen}
\tableofcontents
\end{screen}\vspace{3mm}
\end{quote}}


\newpage
\pagestyle{fancy}
\section{論文構成の掟}\label{ux8ad6ux6587ux69cbux6210ux306eux639f}

\subsection{全体の構成}\label{ux5168ux4f53ux306eux69cbux6210}

以下が論文の構成例。

\begin{enumerate}
\tightlist
\item
  目次
\item
  概要 (``Abstract'',必要に応じて)
\item
  はじめに(「序論」, ``Introduction'')
\item
  研究方法 (``Methods and Materials'')
\item
  結果および考察 (``Results and Discussion'')
\item
  まとめ(「結論」,``Conclusion'')
\item
  謝辞 (``Acknowledgement'', 必要に応じて)
\item
  付録 (``Appendix'', 必要に応じて)
\end{enumerate}

\subsection{「概要 (Abstract)」}\label{ux6982ux8981-abstract}

論文提出の際には概要の提出も要求されることが多い。

\begin{itemize}
\tightlist
\item
  概要は全体の要約なので, (1) 何を目的として, (2) 何を行い,
  (3)どういう結果を得たかを簡潔に述べる。
\item
  分量は(1)(2)(3)=3:3:4程度を目安に。(1)が半分以上占めると,内容がうすく感じられて良くない。
\item
  先行研究にふれることは少ない。
\end{itemize}

\subsection{「はじめに
(Introduction)」}\label{ux306fux3058ux3081ux306b-introduction}

「はじめに」では,\textbf{研究の背景}(本研究の動機・必要性,世の中でされてる関
連研究等), \textbf{問題提起},\textbf{研究の目的}について書く。
\textbf{論文の存在意義が問われる部分}である。
読者に興味をもってもらえるようしっかり書くこと。

なお,「はじめに」は「まえがき」とは違う。 「まえがき」は,
本のはじめに執筆のきっかけになったよもやま話等を書く部分で,論文では「まえがき」は書かない。

\subsubsection{「はじめに」の構成例}\label{ux306fux3058ux3081ux306bux306eux69cbux6210ux4f8b}

\begin{enumerate}
\tightlist
\item
  \textbf{大きな目的}:
  自分の研究が何を実現したり解明したりすることを目指すものか,すなわち研究の大きな目的を,できるだけ始めの方に書く.いつまでも論文の目的がわからないような書き方はダメ。
\item
  \textbf{必要性}: 大目的を実現する必要性を述べる。
\item
  \textbf{問題提起}:

  \begin{itemize}
  \tightlist
  \item
    大目的を達成するために何をする必要があるのか,そのためにどのような問題があるかのを述べる。この問題提起によって,\textbf{読者の関心を自分の研究トピックに誘導する}。
  \item
    \textbf{ここまではなるべく簡潔に}!
  \end{itemize}
\item
  \textbf{事例紹介}:
  提起した問題を解決するための従来研究を紹介しながら列挙する。自分と同じ(もしくは似た)目的のものを詳しく列挙する事で,研究の位置づけを明確にする。
\item
  \textbf{小目的}:
  問題解決のために自分が\textbf{(研究で取り組んだ具体的な目的)}を明記する。

  \begin{itemize}
  \tightlist
  \item
    検証する\textbf{仮説}や,\textbf{達成すべき目標}を明記する。
  \item
    具体例

    \begin{enumerate}
    \tightlist
    \item
      大目的: 火星ロケットを作る!
    \item
      必要性: 地球にはもう資源が乏しいが,火星にはたくさんの資源がある!
    \item
      問題提起: 火星ロケットを作るには,より軽量で丈夫な金属である。
    \item
      事例紹介: ○○らは△△合金の開発を行ったがこれはやや熱に弱い
      という欠点がある。一方で,□□らは##合金を\ldots{}.が,これは熱に強
      い一方で固さが不十分であった。
      しかし,この両者の長所をあわせもつ合金を開発できれば\ldots{}。
    \item
      小目的: 本研究の目的は,○○に着目する事で,軽量かつ丈夫で熱
      にも強い金属を開発することである。
    \item
      本論文の構成は以下の通りである。第2章では\ldots{}について述べ, 第
      3章でその結果の考察を行う。第4章では\ldots{}
    \end{enumerate}
  \item
    悪い目的の例:
    「本研究の目的は,日照量と収穫量の関係を調べることである」

    \begin{itemize}
    \tightlist
    \item
      この文では手法しか述べておらず,肝心の目的(ゴール設定)が書かれていない(両者の関係を調べることで,何を明らかにしたいのか書かれていない)。
    \end{itemize}
  \end{itemize}
\end{enumerate}

\subsection{「研究方法 (Materials and
Methods)」}\label{ux7814ux7a76ux65b9ux6cd5-materials-and-methods}

\begin{itemize}
\tightlist
\item
  「◯◯の解析方法」等,もっと具体的なタイトルをつけることもある。また,
\item
  「実験方法」や「解析方法」を具体的に述べる。
\item
  第3者がこれを読んで同じ実験や解析を再現できるだけの情報を書くこと。
\item
  一般的な解析方法や計測方法については,その名称を書けばよく,具体的なプログラミング方法等や,計測方法の細かい手順の解説は不要。
\end{itemize}

\subsection{「結果および考察 (Results and
Discussion)」}\label{ux7d50ux679cux304aux3088ux3073ux8003ux5bdf-results-and-discussion}

\begin{itemize}
\tightlist
\item
  「結果」と「考察」のそれぞれを独立の章や節にする場合もある。
\item
  複数の実験等を行ったときには,「研究方法」と「結果と考察」を実験毎に一つの章にまとめることもある。
\end{itemize}

\subsubsection{「結果」}\label{ux7d50ux679c}

\begin{itemize}
\tightlist
\item
  実験の結果,データから読み取れる客観的事実を説明する。
\item
  ただし,データやグラフを並べて,その説明をすれば良いというものではない。\textbf{この論文の目的を達成するために,どのような図表を用意して,何に着目して説明をするのが良いか,よく計画を練ってから書く}。言い換えると,読者が結果の説明を読み進めるうちに,論文の目的に対する答えを自然と把握できるような書き方が良い。
\end{itemize}

\subsubsection{「考察」}\label{ux8003ux5bdf}

考察の目的には以下の2点がある。

\begin{enumerate}
\tightlist
\item
  本論文における目的に即した結論を導く。

  \begin{itemize}
  \tightlist
  \item
    本実験結果と他の研究(文献)との関連を説明する。つまり,本実験結果が関連する他の知見と合致するのか,それとも矛盾するのかを議論する。これにより,本結果を一般化したどのような結論を導き出せるかを,論文の目的に即して説明する。
  \item
    実験結果の妥当性を説明する。もし,実験手法になんらかの問題があると考える場合には,その問題点が結果にいかに影響した可能性があるかを説明した上で,結果の妥当性を議論する。また,妥当性の議論においても,他の研究との関連による説明は重要。
  \end{itemize}
\item
  本研究の重要性を説明する。言い換えると,本研究結果から導かれる重要な問題を提起する。

  \begin{itemize}
  \tightlist
  \item
    本実験結果がなぜ生じたか,その理由を考察する(理由に関する仮説形成)。
  \item
    本実験結果を認めると,さらにどのような現象の予測や応用可能性があるかを考察する(今後の課題に関連する仮説形成)。
  \end{itemize}
\end{enumerate}

なお,「考察」を独立な章にまとめた時には,結果についての簡単なまとめの段落をまず設けること。

\subsection{「まとめ
(Conclusion)」}\label{ux307eux3068ux3081-conclusion}

「まとめ」に何を書くかは,論文誌の種類によって異なる。短い報告では「まとめ」を書かない場合もあるが,長い報告では研究の重要性を強調するために「まとめ」を設ける。

\begin{itemize}
\tightlist
\item
  何を行い,その結果,研究目的に対してどのような結果が得られたかを具体的に書く。
\item
  結果に関する面白い点を具体的に列挙する。
\item
  今後の課題(将来の展望)を述べる。ただし,「考察」で十分議論をしているときには重複しないようにする。具体的には以下の議論がありえる。

  \begin{enumerate}
  \tightlist
  \item
    考察で得られた新仮説の証明方法に関する補足をする。
  \item
    得られた結果を認めた場合に,さらに発展的に考えられる課題を提案する。
  \item
    ありきたりなこと(サンプル数を増やしたい等)は,書き手にとっても読者にとっても面白くない(重要な情報にならない)ので,あまりスペースを割かないこと。
  \end{enumerate}
\end{itemize}

\subsection{\texorpdfstring{「謝辞(``Acknowledgement'',
章/節番号無し)」}{「謝辞(Acknowledgement, 章/節番号無し)」}}\label{ux8b1dux8f9eacknowledgement-ux7ae0ux7bc0ux756aux53f7ux7121ux3057}

\begin{itemize}
\tightlist
\item
  研究の推進のために助成を受けた予算や,有用なアドバイスをいただいた人に対する謝辞を記載する。
\item
  人名は必ずフルネームで書く。
\end{itemize}

\subsection{\texorpdfstring{「参考文献(``Bibliography'',
章/節番号無し)」}{「参考文献(Bibliography, 章/節番号無し)」}}\label{ux53c2ux8003ux6587ux732ebibliography-ux7ae0ux7bc0ux756aux53f7ux7121ux3057}

\begin{enumerate}
\item
  論文中で, 既にわかっている事実を述べようとするときには,
  その事実が明記さ れている参考文献を必ず引用する。
  自分の意見か他の人の意見かわからない書き方はしてはいけない。
\item
  引用した文献のリストを論文の最後につくる。
\item
  文献リストには以下の情報が必要。

  \begin{enumerate}
  \item
    論文の場合

    \begin{quote}
    著者, 論文名, 雑誌, 巻, ページ, 発行年
    \end{quote}

    例) D. F. Hoyt, C. R. Taylor, ``Gait and the energetics of
    locomotion in horses,'' Nature, 292(16), 238-240, 1981
  \item
    本の一部の場合

    \begin{quote}
    著者, 章名, In: 編集者名(ed) 本の名前, 出版社名, ページ, 発行年
    \end{quote}

    例) Cohen AH, ``Evolution of the vertebrate central pattern
    generator for locomotion'', In: Cohen AH, Rossignol S, Grillner S
    (eds) Neural control of rhythmic movements in vertebrates, John
    Wiley and Sons, pp.~129-166, 1988
  \item
    著者が3人以上の場合には,C. R. Taylor et al. といった具合に最
    初の著者のみを代表にして書くことが多い。
  \end{enumerate}

  各項目の間は原則として'',''で区切る。
\item
  文献リストのフォーマット(上記情報の掲載順)のお流儀はいろいろある
  が,どれかで統一する。
\item
  参考文献には参照番号をつけて, それにより本文中で引用する。
  LaTeXなどでは文献をデータベース化しておけば, 本文中での引用に応じて,
  適当なフォーマットに直した文献リストを作成してくれ,参照番号も自動でつ
  けてくれる。
  参照番号をつけずに、文章中では著者名(年)で参照する方法もある。

  \begin{enumerate}
  \item
    文献リストに参照番号をつけている場合の引用例

    \begin{enumerate}
    \item
      例

      \begin{quote}
      ○○は××であることが実験により確認され{[}1{]}, その理論的証明は△△ら
      によってなされた{[}2,3{]}。また,☆☆らは□□を報告している{[}4{]}.
      \end{quote}
    \item
      複数の文献を引用する場合(LaTeXの場合はcitesスタイルファイル
      を使うと便利)

      \begin{quote}
      ×: △△は□□であることが報告されている{[}1{]},{[}2{]}\\
      ○: △△は□□であることが報告されている{[}1{]}{[}2{]}\\
      ○: △△は□□であることが報告されている{[}1,2{]}
      \end{quote}
    \item
      句読点との順番を逆にしてはいけない。

      \begin{quote}
      \textbf{間違った例)}○○は××であることが実験により確認され,{[}1{]}
      その理論的証明は△△らによってなされた。 {[}2,3{]}
      \end{quote}
    \end{enumerate}
  \item
    文献リストに参照番号をつけない場合の引用例

    \begin{quote}
    例) ○○ら(1997)に××であることが報告され, その理論的証明は△△ら
    (2003)によってなされた。さらに◎◎についても多くの報告がある
    (□□1993,▽▽1995,☆☆ら2002)
    \end{quote}
  \end{enumerate}
\end{enumerate}

\subsection{\texorpdfstring{「付録」 (``Appendix'',
必要に応じて)}{「付録」 (Appendix, 必要に応じて)}}\label{ux4ed8ux9332-appendix-ux5fc5ux8981ux306bux5fdcux3058ux3066}

\begin{itemize}
\tightlist
\item
  本文中での,本筋を明確にするために,細かくかつ紙面をとるような内容は付録にすることが多い。例えば,本文中の数式の証明等の細かい計算過程,本文で特に述べる必要の無いコメント的なことや技術的なことをここに書く。
  - 短い注釈等はここに入れずに, 脚注にすべき。

  \begin{itemize}
  \tightlist
  \item
    付録をつくるときには, 必ず本文中の関連場所で引用すること。
    (「詳しくは付録A参照のこと」等。)
  \item
    注)付録の位置は、謝辞の前に置く場合、謝辞の後ろに置く場合,参考文献の後におくといろいろなお流儀がある。
  \end{itemize}
\end{itemize}

\section{文章の書き方のお作法}\label{ux6587ux7ae0ux306eux66f8ux304dux65b9ux306eux304aux4f5cux6cd5}

\subsection{論文の掟}\label{ux8ad6ux6587ux306eux639f}

\begin{enumerate}
\tightlist
\item
  「はじめに」に述べる\textbf{目的に対応した結果}をきちんと述べる こと。
\item
  \textbf{目的は少なくとも3回登場} する。

  \begin{enumerate}
  \tightlist
  \item
    「はじめに」に書く。
  \item
    結果を示すとき,
    それが目的に対応した結果であることをきちんとわかるように書く。
  \item
    最後のまとめで, 目的が達成されたかどうかがはっきりわかるように書く。
  \end{enumerate}
\item
  書いてる内容が, 自分の意見か誰かの意見によるものかを明らかにする。
  誰かのデータや結果等を引用するときには必ず引用元になる参考文献を明記
  する。
\end{enumerate}

\subsection{文章の書き方の掟}\label{ux6587ux7ae0ux306eux66f8ux304dux65b9ux306eux639f}

\begin{enumerate}
\tightlist
\item
  何の話をしているか常に明確になるようにする。
\item
  大きな話(目的や結論)から書き,
  徐々に小さい話(詳細な説明)にうつるのが原則。
\item
  「事実」を書き, 次に「考察」を書くのを繰り返すのが基本。

  \begin{enumerate}
  \tightlist
  \item
    事実だか考察だかわからないような書き方はダメ。
  \item
    複数の事実を並べてから, それぞれの考察に入る場合には,
    各事実項目に番号をふって,
    それを引用しながら考察をすると分かりやすくなることが多い。
  \end{enumerate}
\item
  行間を読ませるような書き方はダメ。
  読者は何も考えなくても読むだけでわかるように書く。
\item
  一段落に一話題。

  \begin{enumerate}
  \tightlist
  \item
    5,6行〜10行程度を分割の目安に。
  \item
    1文で1段落は不可
  \end{enumerate}
\item
  「それ」などの指示代名詞は、原則として前文までに指すものが明確にないとダメ。
\item
  箇条書きについて

  \begin{enumerate}
  \tightlist
  \item
    短い文を列挙したい時に使う。長い文の列挙のときには,段
    落分けやセクション分けを上手に使う。
  \item
    箇条書は文章中で少し情報をまとめるためのものであり,
    あるセクションが箇条書のみということは, 論文ではあり得ない。
    箇条書を使うときには, それが何についての説明かをまず書く。
  \end{enumerate}
\end{enumerate}

\subsection{文の書き方の掟}\label{ux6587ux306eux66f8ux304dux65b9ux306eux639f}

わかりやすく誤解の無い日本語を書くことに尽きる。

\begin{enumerate}
\item
  文体は「だ、である」調を使う。
\item
  一つの文では一つの事柄のみにふれる(\textbf{一文一義})。
\item
  主語述語を明確に。
  主語と述語が適切に対応していることをよく確認すること。
\item
  受動態はあまり使わないこと。使う場合でも、一つの文中で能動態と受動態を混ぜて使ってはいけない。

  \begin{quote}
  × 脊椎動物は哺乳類を含み、ヒトは哺乳類に含まれる。
  ◯ 脊椎動物は哺乳類を含み、哺乳類はヒトを含む。
  \end{quote}
\item
  修飾する語は, 修飾される語のできるだけ近くにおく。

  \begin{quote}
  × 「大きな論文の流れは以下の通り。 」\ldots{}
  「大きな論文」ってなに?\\
  ○ 「論文の大きな流れは以下の通り。 」
  \end{quote}
\item
  一つの言葉に2つ修飾語が付くときには, 短いほうを修飾される語に近付
  ける。

  \begin{quote}
  × 黒いお腹をすかした猫\\
  ○ お腹をすかした黒い猫
  \end{quote}
\item
  句読点を適切に。 言葉の修飾関係が明確にするように使う。
\item
  接続語, 指示語は乱用しない。 必要最小限にとどめる。 (そして,そこで,
  しかし, これを, それは,\ldots{})
\item
  あいまいな表現が無いか十分気を付ける。

  \begin{itemize}
  \item
    ×「速度が速くなることがわかる。」\ldots{}
    因果関係が不明、何の速度かも不明\\
    ○「△△が大きくなるにつれて, ○○の速度が大きくなることがわかる。」
  \item
    ×「馬は最適な歩行パターンを選んでいることがわかる。」\ldots{}
    どういう意味で 最適か不明\\
    ○「馬はエネルギー効率に関して最適な歩行パターンを選んでいることがわかる。」
  \end{itemize}
\item
  理系の学術的な文章での句読点は''、 。'' ではなく'', 。''や '',
  .''を使うことが多い。\textbf{いずれかの形式で統一する事。}
\end{enumerate}

\subsection{図表の説明の掟}\label{ux56f3ux8868ux306eux8aacux660eux306eux639f}

\begin{enumerate}
\item
  説明にはできるだけ図表を利用する。 イラスト等もたくさん使って良い。
\item
  図表は\textbf{無駄には多くしない}こと。
  特に実験結果等で似たグラフが多く得られたような場合は, 1つのグラフに
  まとめたり,平均値をグラフにしたり等の工夫をする。
\item
  図表はページの上か下におき,\textbf{文章中には挿入しないこと}。
  LaTeXの位置指定なら\lstinline![tbp]!となる。 図表は文の一部ではなく,
  説明のための補助であることに注意。
\item
  キャプション

  \begin{enumerate}
  \tightlist
  \item
    図表にはcaption(\textbf{タイトル}と\textbf{短い説明})をつける。

    \begin{enumerate}
    \tightlist
    \item
      図の説明(caption)は図の下に。
    \item
      表の説明(caption)は表の上に。
    \end{enumerate}
  \item
    タイトルには,図表が何を示すものかを簡潔にまとめる。
  \item
    細かい説明はタイトルの後に書く。
    文を読まずに図表とそのキャプションのみを見て, 粗筋がだいたい分かるよ
    うに図表と説明があることが望ましい。
    キャプションに書く説明は,本文と多少重複してもかまわないが,そのとき
    は簡潔に要約する。
  \end{enumerate}
\item
  他の文献から引用した図表は, 必ず脚注に文献を引用する

  \begin{quote}
  例1:''○○○○(1998)より''\\
  例2:''○○○○より{[}3{]}''
  \end{quote}
\item
  図表は\textbf{必ず本文中で登場順に引用して説明する}
\item
  本文中での各図表の説明は, 以下の順が原則

  \begin{enumerate}
  \tightlist
  \item
    図表が\textbf{何を表すものか}
  \item
    図表を\textbf{見てわかること}
  \item
    結果の\textbf{考察}(結果が生じた原因の推定)
  \end{enumerate}

  \begin{quote}
  例)図3に2月1日の気温の時間変化の様子を示す。
  早朝には氷点下にまで気温が
  下がっているが日中の最高気温は3月並にまで上昇したことがわかる。
  この日は高気圧に覆われたため, 朝は放射冷却によって冷え込んだが,
  晴天により日中の気温が上昇したことが原因と考えられる。
  \end{quote}
\end{enumerate}

\subsection{グラフ作成の掟}\label{ux30b0ux30e9ux30d5ux4f5cux6210ux306eux639f}

\begin{enumerate}
\tightlist
\item
  \textbf{各軸のタイトル} と\textbf{単位} を忘れずに。
\item
  グラフの縦軸と横軸は, グラフの特徴を有効に示せる範囲を選ぶこと。
\item
  複数のグラフを比較して見るような場合には, それぞれの縦軸と横軸の
  範囲をそろえるほうが(多くの場合)良い。
\end{enumerate}

\subsection{数式の書き方の掟}\label{ux6570ux5f0fux306eux66f8ux304dux65b9ux306eux639f}

\begin{enumerate}
\item
  数式は文章の一部である。よって数式のあとには\textbf{必要に応じて句読点}を書く。
\item
  \textbf{変数を表す記号はイタリック体} にする。(TeXなら\$で囲む)
\item
  数式で用いる記号は, その数式の前か直後に必ず説明をする。

  \begin{quote}
  例) ニュートンの運動方程式は次式で与えられる。

  \[m\boldsymbol{a}=\boldsymbol{F}.\noindent\]

  ここで, \(m\) は質点の質量, \(\boldsymbol{a}\)は質点の加速度ベクトル,
  \(\boldsymbol{F}\)は質点に加えられた力ベクトルである。
  \end{quote}
\item
  \(\sin\), \(\cos\), \(\exp\)等の \textbf{特殊関数はローマン体}
  にする。

  \begin{quote}
  × \(sin x\): これでは,どこまでが変数かわからなくなる。\\
  ○ \(\sin x\): TeXの数式モードでは \lstinline!\sin!と書く。
  \end{quote}
\item
  物理量をあらわす数値には必ず単位をつける。 \textbf{単位はローマン体}
  にする。

  \begin{quote}
  例)ロボットの胴体の質量\(M\)を\(10\) kg, 足の長さ\(l\)を\(0.5\) m とし
  て計算を行なった。
  \end{quote}

  ここで, 数値と単位の間にスペースが入ってるのに注意。
  変数記号に単位をつける時には,変数と単位の区別がつきやすいように{[}{]}や()で囲む。

  \begin{quote}
  例) \(m\) {[}kg{]}, \(t\) {[}s{]}
  \end{quote}
\item
  数式を変形する時には、どのように変形したか、そのプロセスが明確にわかるように書く。
\end{enumerate}

\subsection{空白(スペース)の掟}\label{ux7a7aux767dux30b9ux30daux30fcux30b9ux306eux639f}

以下の場合,空白(スペース)が必要

\begin{itemize}
\tightlist
\item
  句読点(``,'', ``.'', ``;'', ``:'')の直後
\item
  省略記号``.''の直後
\end{itemize}

\begin{quote}
例) J. S. Bach (Johann Sebastian Bachのイニシャル)
\end{quote}

\begin{itemize}
\tightlist
\item
  数値と単位の間
\item
  数学記号と単位の間
\end{itemize}

ただし,以下の場合は空白不要 - ラテン語の略語

\begin{quote}
例) e.g., i.e.
\end{quote}

\begin{itemize}
\tightlist
\item
  比率を表す``:''
\end{itemize}

\begin{quote}
例) 塩の質量:水の質量=1:10
\end{quote}

\subsection{その他}\label{ux305dux306eux4ed6}

\begin{itemize}
\tightlist
\item
  章毎には改ページを行うが、節単位では改頁しない。(LaTeXは自動で改ページ等を行ってくれるので、余計なことはしないほうが身のため)
\item
  被験者のデータを示すときには,被験者A,B,\ldots{}等と書く。
\item
  LaTeXの改行コマンド(\lstinline!\\!)は原則として利用禁止
\item
  半角``.'', ``,''の直後には半角スペースを入れる
\end{itemize}

\subsection{文章書き最大の掟}\label{ux6587ux7ae0ux66f8ux304dux6700ux5927ux306eux639f}

まずは,最善を尽くした初校をつくる。
その後,書いた文章は\textbf{適当な時間をおいて}何度も読み直して直す。
完成と思った後に卒論程度の文章量なら100回程度は読み直す。
校正は確率的にしかできない。
間違い検出の精度をあげるためには校正を何度も繰り返すしかない。

\section{練習}\label{ux7df4ux7fd2}

以下の文は, どういう点で悪いか? また, どのように改善したらよいか考えよ。

\begin{enumerate}
\item
  「図2は, 3月1日の気温の時間変化を示したグラフである。 これより, 気
  温は上昇することがわかる」
\item
  「移動速度が速くなると足を動かす周波数は高くなる。
  移動速度が遅くなっても足の振り幅はあまりかわらない。
  移動速度が速くなっても足が地面からはなれてる時間はあまりかわらない。
  」
\end{enumerate}

\section*{参考資料}\label{ux305dux306eux4ed6ux53c2ux8003ux306bux306aux308bux8cc7ux6599}
\addcontentsline{toc}{section}{参考資料}

\begin{itemize}
\tightlist
\item
  「APA
  論文作成マニュアル(第二版)」,アメリカ心理学会(APA)著,前田,江藤,田中(訳),医学書店,2011
\item
  「理科系の作文技術」,木下是雄著,中公新書,1981
\item
  「例題で学ぶ原稿の書き方ーわかりやすい文章のために」,八木和久著,米田出版,2001
\item
  「大学生の論文執筆法」,石原千秋著,筑摩新書,2006
\end{itemize}

\section{このドキュメントの著作権について}\label{ux3053ux306eux30c9ux30adux30e5ux30e1ux30f3ux30c8ux306eux8457ux4f5cux6a29ux306bux3064ux3044ux3066}

\begin{enumerate}
\item
  本稿の著作権は西井淳\url{nishii@sci.yamaguchi-u.ac.jp}が有します。
\item
  非商用目的での複製は許可しますが、修正を加えた場合は必ず修正点および加筆者の
  氏名・連絡先、修正した日付を明記してください。また本著作権表示の削除は行っ
  てはいけません。
\item
  本稿に含まれている間違い等によりなんらかの被害を被ったとしても著者は一切
  責任を負いません。
\end{enumerate}

間違い等の連絡や加筆修正要望等の連絡は大歓迎です。

\end{document}
