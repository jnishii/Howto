\newif\ifkanjifonts
\kanjifontstrue

\documentclass[12pt]{jarticle}
\usepackage{kanjifonts}
\usepackage[height=26cm,width=16cm]{geometry}
\usepackage[hypertex]{hyperref}
\usepackage{ascmac}
\usepackage{amsmath}

\def\puttitle#1{\ifkanjifonts{\Huge\lxpopfamily #1}\else{\Huge\gtfamily #1}\fi}

\begin{document}
\noindent
\rule{\linewidth}{1pt}
\begin{center}
\puttitle{レポート・答案の書き方の掟}
\end{center}
\begin{flushright}
\today 版\hspace{.5cm} 西井 淳
\end{flushright}
\rule{\linewidth}{1pt} 
{\small
\begin{quote}
\begin{screen}
\tableofcontents
\end{screen}
\end{quote}}


\section{最大の掟}

ヒトの脳と計算機にはどういう違いがあるだろうか?
計算機が進歩したらヒトの脳のようなものができるだろうか?
このような質問をすると、多くのヒトが「計算機は与えられた命令を論理的に
処理するだけであって、ヒトにあるような創造性や解くべき問題をみずから考
える力はない」といった答をするヒトが多い。
では、計算機のもっていないという何かをヒトは本当に持っているのだろうか?

レポートや答案とは、ヒトとしての知能を解答者がもっているかどう
かを確認するための一種の知能テストである。以下に例を示そう。
\begin{itemize}
\item レポートは「本」や「黒板」に書いてあったことをそのまま写したよう
  なものであった。 
  若干高級なものには「…が興味深かった」等の接尾語がついているものも
  あったが。。。
  \begin{quote}
    解答者はひょっとしてヒトではなく、単なる複写機やメモリ、検索エンジ
    ンの類なのだろうか? 
  \end{quote}
\item 提出されたレポートはどれも同じ様な内容であった。
  \begin{quote}
    解答者はヒトではなく、決まりきった入出力関係をプログラムされたロボッ
    トたちだろうか?
  \end{quote}
\item 提出されたレポートは論理性や創造性もないものであった。
  (「ヒトの脳と計算機にはどういう違いがあるだろうか?」という問に対し
  ては、みんな同じ様に「計算機は与えられた命令を論理的に
  処理するだけであって、ヒトにあるような創造性や解くべき問題をみずから考
  える力はない」と理由もなくかいてあった。)
  \begin{quote}
    レポートを出力をしたのは壊れたコンピュータなのだろうか?
  \end{quote}
\end{itemize}
要は、レポートや答案を書くということは、
解答者が、単なる機械ではなく、知性をもったヒトであることを
示せるかどうかを試されているわけである。
よって、レポート・答案作成における最大の掟は
\textbf{自分がヒトであるという証明をできるよう努力すること}である。

\section{文章書きの掟}

解答者が現在の計算機以上の存在であることを示すには、
少なくとも以下をよく吟味して示す必要がある。

\begin{enumerate}
\item 「タイトル」内容をよく表すタイトルをつけているか?
\item 「目的」:目的(論点)がはっきりしているか?
\item 「論理性」:文章が論理的であるか?
\item 「主張」:単なる事実の列挙ではなく、自分の意見を述べているか?
\item 「構成」:文章全体の流れ(起承転結)がまとまっているか?
\item 「伝達手段」:日本語は適切か?正確に内容を伝えられる表現になっているか?
\item 「情報源」:他人の文やデータを無断借用していないか?
\item 「独創性」:あなたしか書けない文章であるか?
\end{enumerate}
各項目について、以下にもう少し詳しく説明をする。

\subsection{「タイトル」:内容をよく表すタイトルをつけているか?}

具体的なタイトルが与えられているときには、それに従えばよいが、
そうでない時には、見ただけでそのレポートの目的や概要が
わかるようなタイトルを工夫してつけること。

例えば「ロボットと人間」といったタイトルでは「ロボットと人間の物理的身
体構造の違い」なのか、「現在のロボットと人間の共存する社会における問題」
なのか、具体的なことが全然わからない。
新聞やニュースのタイトルのように見ただけで注意をひくようなものを考える
こと。
タイトルの善し悪しで、中身の注目度が変わります。


\subsection{「目的」:目的(論点)がはっきりしているか?}

文章のできるだけはじめのほうに目的を明確に書く。
\begin{quote}
  例)本レポートの目的は○○を明らかにすることである。
\end{quote}
決してぼかした書き方はしてはいけない。
目的のないレポートは論外である。

\subsection{「論理性」:文章が論理的であるか?}

読み手に「なんでやねん」とつっこみを入れられるような部分を作ってはいけ
ない。逆に書き手は自分自身にいつも「なんでやねん」とつっこみを入れなが
ら論理的なつながりに問題がないか十分気をつけること。

悪い例「計算機は機械なので感情をもつことはできない」

\subsection{「主張」:単なる事実の列挙ではなく、自分の意見を述べているか?}

調べたことをただ書くだけでは、コピー機や検索エンジンの域を出るものでは
ない。必ず自分の視点・主張をまじえた考察等をしっかりのべること。
知識を並べるだけなら検索エンジンでも出来る。
画家がいくら絵の具の種類をたくさん知っていても描きたいものがなければ何
の役にもたたない。
感動したこと伝えたいことが本当にあれば、言葉のたどたどしい子供でもしっ
かり伝えるものである。

ただし、単なる感想の羅列になってはいけない。例えば、「○○を見た。楽し
かった。」でおわってしまうのでは、幼稚園の子供のほほえましい作文になっ
てしまう。しかしこの後に、どういう点がどのように、また、なぜ楽しかった
か、その理由が詳しく具体的に説明されていれば、説得力のある文に変わる。

\subsection{「伝達手段」:日本語は適切か?正確に内容を伝えられる表現になっているか?}

読み手が、日本語の解読に苦労するような文を書いてはいけない。
自分でわかると思った文でも、まったく違う環境で生まれ育ったヒトにはまる
でわからないことはよくある。
「このごらいでわかるだろう」という妥協をしてはいけない。
常に正確に内容が相手に伝わるかを自問自答し続けること。

例えば、友達と一日野山をかけめぐって、夕方に「つかれたな」と言えば、
同じ経験を共有するその友達には「一日走入まわってたからなー」と通じるだ
ろうが、その場にいないヒトに電話して「つかれた」とだけ言っても、
「遊び付かれた」のか、「人生につかれた」のか、はたまた「悪霊にとりつか
れたのか」さっぱりわからない。
話し手にとって当り前のことも、きちんと言わないと相手には通じないのが普
通である。

\subsection{「構成」:文章全体の流れ(起承転結)がまとまっているか?}

書いた文の内容を正確に伝えるには、1つひとつの文レベルで正確に書く一方、
目的と結論を明確にし、そして目的から結論を出すまでの論理を示すために、
文章全体の構成をしっかり考える必要がある。

\subsection{「情報源」:他人の文やデータを無断借用していないか?}

他人の書いたものを無断借用するのは、所謂著作権法違反であり、そもそもマ
ナー違反である。 
また、レポートでいきなり数値データを出しても、どうやって求めたか、
どういう文献にあったかという出どころを明確にしないと、信憑性の無い数字
ということになる。
他人の文やデータを参考にしたところには、必ずその出どころを明記すること。

\subsection{「独創性」:あなたしか書けない文章であるか?}

たとえヒトの文章を丸写ししたのでなかったとしても、
内容が他人とよく似ていたら、そのレポートには独創性が無いと言うことにな
る。他人の思い付かないことを提案できる独創性が、社会に出てから常に要求
されることになる。

企業であれば、新しく開発したつもりの技術を使って製品をつくったとき、他
の企業がすでに特許を持っていれば、特許侵害によって賠償金を請求される。
研究者が他人と同じ内容の論文を発表しても、誰にも相手にされず、ただの時
間の浪費ということになるか、下手をすれば「盗作」とされ、社会的信頼を失
う。

本当に世界で自分だけの着目点・主張等が書かれているかよく考えること。

\section{最大の掟もう一度}

繰り返しであるが、レポートや答案を書く上で最も大事なことは、
あなたという知性あるヒトの存在を示すことである。
ヒトの書いたもの(調べたこと)を単に写して出すなら、それは、あなたの存在
が無意味であるということを示すことになる。

有名な画家の絵には、必ず特有の特徴がある。ゴッホの絵にはゴッホの特徴が
にじみでていて、はじめて見る絵でも大抵ゴッホの絵とわかるものである。
同様に書いたものには必ずそのヒトの個性が表れるはずである。
もしそのヒトの個性が表れていないなら、その文はそのヒトの書いたものとは
言えない。
文章を書くとは、そのあなただけの何かを見付け、それを文字によって十分ヒ
トに伝えるにはどうすべきかいう創意工夫を行うことである。


\section{この文書の著作権について}

\begin{enumerate}
\item 本稿の著作権は西井淳\url{nishii@sci.yamaguchi-u.ac.jp}が有します。
\item 非商用目的での複製は許可しますが、修正を加えた場合は必ず修正点および加筆者の
氏名・連絡先、修正した日付を明記してください。また本著作権表示の削除は行っ
てはいけません。
\item 本稿に含まれている間違い等によりなんらかの被害を被ったとしても著者は一切
責任を負いません。
\end{enumerate}

\end{document}
