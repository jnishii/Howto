\documentclass[11pt]{jarticle}

%\usepackage{kanjifonts}
\usepackage{amsmath}
\usepackage{ascmac}
\usepackage{geometry}
\usepackage{fancyhdr}
\geometry{body={160mm,230mm}}

%% \renewcommand{\labelitemi}{一.}
%% \renewcommand{\labelitemi}{□ }
%% \renewcommand{\labelitemii}{□ }
\usepackage{latexsym}
\renewcommand{\labelitemi}{$\Box$}
\renewcommand{\labelitemii}{$\Box$}
\renewcommand{\labelitemiii}{- }


\begin{document}

\pagestyle{empty}
%\noindent\rule{\linewidth}{0.3pt}
\begin{center}
  {\LARGE 卒論確認事項}
\end{center}
\begin{flushright}
%{\popfamily\today 版}
{\today 版}
\end{flushright}
%\rule{\linewidth}{0.3pt} 

\begin{itemize}
\item 名前・連絡先(住所・電話・メール)確認
\item アカウント名確認
\item 大学院進学・就職希望確認
\item 英語は得意?
\item 卒論修論発表会について
  \begin{itemize}
  \item 修論発表会:2/13(月)
  \item 卒論発表会:2/20(月)
  \end{itemize}
\item 宿題
  \begin{itemize}
  \item UNIX/LINUXの基本操作($*$印がついてないセクションを読む)
  \item UNIX演習
  \item C演習
%   \item \verb|http://www.iknow.co.jp/|で各自英語の勉強
%   \item 英語の教科書購入(Amazon等で各自買ってください。リンクは西井のホームページ参照)
%     \begin{enumerate}
%     \item フォニックス"発音"トレーニングBook, ジュミック今井, 明日香出版社
%     \item 他にも...(現在考え中)
%     \end{enumerate}
%   \item  数学・物理に自信のない人は西井のホームページの
%     「お手製テキスト集」にある\textbf{「数学の基礎訓練(基礎編)」}と
%     \textbf{「力学の基礎訓練」}を勉強しておく
%  \item いろいろと情報収集・お勉強
%    \begin{itemize}
%    \item MIT OpenCourseWare
%    \item TED
%    \end{itemize}
  \end{itemize}
\item 春休みのお勉強(宿題、計測実験兼計算機演習等)
  \begin{itemize}
  \item スケジュール確認(2/27-)
  \end{itemize}
\item 研究室でのデューティ(予定)
  \begin{itemize}
  \item セミナー、本読み、各種お勉強会等
  \item 必要に応じて物理・数学等の講義出席
  \end{itemize}
\item 卒論の生活
  \begin{quote}
	研究室所属後の生活は、普通の社会生活に準じます。
  言い替えると大学の講義期間とは無関係ですので、公式の夏休み等はあり
  ません。休暇は社会生活の常識の範囲内と考えましょう。
  (休むなという意味ではありません。休むときにはしっかり休みましょ
  う)
  \end{quote}
\end{itemize}

\end{document}